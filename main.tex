\documentclass[]{article}

%opening
\title{Equality}
\author{Esteve Tarrago}

\begin{document}

\maketitle

\begin{abstract}
    In this essay I will try to expose my point of view on wealthiness inequalities and how they could be solve.
\end{abstract}

\section{Two opposite models}
\subsection{Capitalism}
"Capitalism is an economic system based on the private ownership of the
means of production and their operation for profit. Characteristics central
to capitalism include private property, capital accumulation, wage labor,
voluntary exchange, a price system, and competitive markets.
In a capitalist market economy, decision-making and investment are
determined by every owner of wealth, property or production ability
in financial and capital markets, whereas prices and the distribution
of goods and services are mainly determined by competition in goods
and services markets." Wikipedia

There ara a few problems with this system:
\begin{enumerate}
    \item \textbf{Born inequalities}: The amount of wealth a person may get is
     strongly correlated on their fathers wealth.
    \item \textbf{Wealth system influence}:
    \item \textbf{People need to work}:
\end{enumerate}

\section{The soccer example}
Imagine a league with N members. The capital of all teams is represented by the vector $
\vec{C}(t)$. It will increase from season to season according to the position
they finished in the previous season.

Lets suppose they all start with the same amount of money $c_o$:
\[ \vec{C}(0) = c_o * \vec{1} \]

Imagine a total amount of money $T$ that will be shared among all members according to their rank.
Each team play against all other teams. The odds of victory are proportional to the capital they have.




\subsection{"Communism"}
Communism is an economic system where the group owns the factors of production. In countries, the government represents the group. The means of production are labor, entrepreneurship, capital goods, and natural resources. Although the labor force aren't legally owned by the government, the central planners tell the people where they should work.

German philosopher Karl Marx developed the theory of communism.

He said it was, "From each according to his ability, to each according to his need." In his view, capitalistic owners would no longer siphon off all the profits. Instead, the proceeds would go to the workers.

\end{document}
